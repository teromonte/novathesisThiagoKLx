%!TEX root = ../template.tex
%%%%%%%%%%%%%%%%%%%%%%%%%%%%%%%%%%%%%%%%%%%%%%%%%%%%%%%%%%%%%%%%%%%
%% chapter1.tex
%% NOVA thesis document file
%%
%% Chapter with introduction
%%%%%%%%%%%%%%%%%%%%%%%%%%%%%%%%%%%%%%%%%%%%%%%%%%%%%%%%%%%%%%%%%%%

\typeout{NT FILE chapter2.tex}%

\chapter{State of the Art}
\label{cha:State of the Art}

\prependtographicspath{{Chapters/Figures/Covers/}}


\section{The Template}
\label{sec:a_bit_of_history}

\begin{figure}[htbp]
  \centering
  \includegraphics[height=1in]{Soviet}
  \caption{Imagem em cons}
  \label{fig:Figuras_Tree_silhouettes-bitmap}
\end{figure}

The KLx system is currently capable of managing IGA processes effectively. However, there are opportunities for improvement in terms of code quality and system capabilities. Code quality can be enhanced by refactoring certain modules to eliminate code smells and improve readability and maintainability. The system’s capabilities can be extended by developing new features and integrations with other applications, as demanded by stakeholders. The implementation of version control with GIT and the automation of application deployment in various environments with GITLAB are also part of the current state of the system.