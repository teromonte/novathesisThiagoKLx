%!TEX root = ../template.tex
%%%%%%%%%%%%%%%%%%%%%%%%%%%%%%%%%%%%%%%%%%%%%%%%%%%%%%%%%%%%%%%%%%%%
%% abstract-pt.tex
%% NOVA thesis document file
%%
%% Abstract in Portuguese
%%%%%%%%%%%%%%%%%%%%%%%%%%%%%%%%%%%%%%%%%%%%%%%%%%%%%%%%%%%%%%%%%%%%

O Sistema de Gestão de Funcionários KLx, AnA, é essencial para automatizar o processo de Governança e Administração de Identidade (IGA) dentro do grupo Credit Agricole, gerenciando mais de 5.000 funcionários. Desenvolvido pela Netwrix e implementado na filial CA Leasing and Factoring em 2018, este sistema tem-se tornado cada vez mais complexo devido à integração de vários departamentos e às contínuas mudanças ambientais. Essa complexidade resultou em desafios de manutenção, problemas de qualidade de código e ineficiências no processo de entrega.

Esta tese aborda esses desafios melhorando a legibilidade e manutenção do código, implementando Git para controle de versão e simplificando processos de implantação usando GitLab. Contribuições significativas incluem a refatoração de módulos principais, o desenvolvimento de uma nova matriz organizacional para perfis e permissões de aplicativos e a integração de novos recursos usando PowerShell, C, XML, JSON e SQL. Esses esforços visam melhorar o desempenho do sistema, facilitar a manutenção futura e garantir a confiabilidade a longo prazo do Sistema de Gestão de Funcionários KLx, AnA.

\keywords{
  Qualidade de Software \and
  Governança e Administração de Identidades \and
  Engenharia Reversa \and
  Refatoração de Código
}
% to add an extra black line
