%!TEX root = ../template.tex
%%%%%%%%%%%%%%%%%%%%%%%%%%%%%%%%%%%%%%%%%%%%%%%%%%%%%%%%%%%%%%%%%%%%
%% abstract-pt.tex
%% NOVA thesis document file
%%
%% Abstract in Portuguese
%%%%%%%%%%%%%%%%%%%%%%%%%%%%%%%%%%%%%%%%%%%%%%%%%%%%%%%%%%%%%%%%%%%%

\typeout{NT FILE abstract-pt.tex}%

O sistema de gestão de funcionários e aplicações KLx, conhecido como AnA, é fundamental para automatizar os processos de Governança e Administração de Identidades (IGA) para mais de 5000 funcionários do Grupo Crédit Agricole. Este software local, desenvolvido pela Netwrix e gerido pela equipa KLx em Lisboa, tornou-se cada vez mais complexo devido à rápida integração de vários departamentos e à evolução dos ambientes. Esta complexidade, aliada à falta de controlo de versões e a processos de implementação ineficientes, levou a uma diminuição da taxa de entrega e da qualidade do código.

O principal objetivo desta investigação é aumentar a eficiência e a eficácia do sistema através da implementação de metodologias de qualidade de software. As soluções propostas incluem a refatoração meticulosa da base de código para melhorar a legibilidade e a manutenibilidade, a implementação do GIT para controlo de versões e a automatização da implementação com o GITLAB. Espera-se que estas intervenções reduzam a duplicação de código, agilizem o processo de entrega e garantam uma base de código robusta e organizada, aumentando assim significativamente a produtividade da equipa e a fiabilidade do sistema.

\keywords{
  Qualidade de Software \and
  Governança e Administração de Identidades \and
  Controlo de Versões \and
  Refatoração de Código
}
% to add an extra black line
