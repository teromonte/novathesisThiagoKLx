%!TEX root = ../template.tex
%%%%%%%%%%%%%%%%%%%%%%%%%%%%%%%%%%%%%%%%%%%%%%%%%%%%%%%%%%%%%%%%%%%%
%% abstract-pt.tex
%% NOVA thesis document file
%%
%% Abstract in Portuguese
%%%%%%%%%%%%%%%%%%%%%%%%%%%%%%%%%%%%%%%%%%%%%%%%%%%%%%%%%%%%%%%%%%%%

\typeout{NT FILE abstract-pt.tex}%

O sistema de gestão de funcionários e aplicativos da KLx, em particular AnA, é de grande auxílio ao automatizar os processos de Governança e Administração de Identidade (IGA) para mais de 5000 utilizadores do Grupo Crédit Agricole. O software local em questão, implementado pela Netwrix e operado pela equipa KLx em Lisboa, tornou-se bastante complicado de gerir devido à integração precoce de um grande número de departamentos e à dinâmica do cenário. Isso, juntamente com a complexidade, estratégias de versionamento ineficazes e más práticas na implantação, levou a uma redução na cadência de entrega e na qualidade do código.

O objetivo básico deste projeto é aumentar a eficiência e eficácia do sistema através da implantação de práticas de qualidade de software. Soluções propostas são a reformulação conservadora do código para melhorar a legibilidade e manutenibilidade, a utilização do GIT para versionamento e a automação da entrega com o GITLAB. Haverá menos duplicação de código, o processo de entrega será aprimorado e o código-base será muito mais durável e limpo com essas intervenções, melhorando assim drasticamente a eficiência da equipe e a confiabilidade do sistema.

\keywords{
  Qualidade de Software \and
  Governança e Administração de Identidades \and
  Controlo de Versões \and
  Refatoração de Código
}
% to add an extra black line
