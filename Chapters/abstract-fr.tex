%!TEX root = ../template.tex
%%%%%%%%%%%%%%%%%%%%%%%%%%%%%%%%%%%%%%%%%%%%%%%%%%%%%%%%%%%%%%%%%%%%
%% abstract-pt.tex
%% NOVA thesis document file
%%
%% Abstract in FR
%%%%%%%%%%%%%%%%%%%%%%%%%%%%%%%%%%%%%%%%%%%%%%%%%%%%%%%%%%%%%%%%%%%%

\typeout{NT FILE abstract-fr.tex}%

Le système de gestion des employés et des applications KLx, connu sous le nom d'AnA, est essentiel pour automatiser les processus de Gouvernance et d'Administration des Identités (IGA) pour plus de 5000 employés du groupe Crédit Agricole. Ce logiciel sur site, développé par Netwrix et géré par l'équipe KLx à Lisbonne, est devenu de plus en plus complexe en raison de l'intégration rapide de divers départements et de l'évolution des environnements. Cette complexité, associée à l'absence de contrôle de version et à des processus de déploiement inefficaces, a entraîné une diminution du taux de livraison et de la qualité du code.

L'objectif principal de cette recherche est d'améliorer l'efficacité et l'efficience du système en mettant en œuvre des méthodologies de qualité logicielle. Les solutions proposées comprennent la refactorisation minutieuse de la base de code pour améliorer la lisibilité et la maintenabilité, la mise en œuvre de GIT pour le contrôle des versions et l'automatisation du déploiement avec GITLAB. Ces interventions devraient réduire la duplication de code, rationaliser le processus de livraison et garantir une base de code robuste et organisée, augmentant ainsi considérablement la productivité de l'équipe et la fiabilité du système.

\keywords{
  Qualité Logicielle \and
  Gouvernance et Administration des Identités \and
  Contrôle de Version \and
  Refactorisation du Code
}
% to add an extra black line
