%!TEX root = ../template.tex
%%%%%%%%%%%%%%%%%%%%%%%%%%%%%%%%%%%%%%%%%%%%%%%%%%%%%%%%%%%%%%%%%%%
%% chapter1.tex
%% NOVA thesis document file
%%
%% Chapter with introduction
%%%%%%%%%%%%%%%%%%%%%%%%%%%%%%%%%%%%%%%%%%%%%%%%%%%%%%%%%%%%%%%%%%%

\typeout{NT FILE chapter1.tex}%

\chapter{Introduction}
\label{cha:introduction}

\prependtographicspath{{Chapters/Figures/Covers/}}


The \gls{novathesis} template was born at the \gls{DI} of  \gls{FCT} of \gls{NOVA}, Portugal. This is a test with citing something~\cite{ecoop12-dias} in the appendix.

\section{Context and Motivation}

The KLx employee and application management system is a crucial tool for automating Identity Governance and Administration (IGA) processes. However, like any software, it requires continuous improvement and extension to meet evolving needs. The motivation for this work stems from the recognition that code quality directly impacts the efficiency and effectiveness of the system. Furthermore, the system’s ability to integrate with other applications within the group is essential for providing a comprehensive IGA solution.

\section{Problem Definition}

The primary problem this thesis addresses is twofold: improving the quality of the KLx system’s code and extending its capabilities to connect with other applications. Code quality will be enhanced by refactoring certain modules to eliminate code smells and improve readability and maintainability. The system’s capabilities will be extended by developing new features and integrations with other applications, as demanded by stakeholders. The implementation of version control with GIT and the automation of application deployment in various environments with GITLAB are also part of the problem definition.

\section{Main Contributions}

This thesis will contribute to the KLx system in several ways. Firstly, it will propose and implement optimizations in the codebase to improve its quality. Secondly, it will extend the system’s capabilities to connect with other applications, enhancing its functionality and utility. Thirdly, it will implement version control and automate deployment processes, thereby improving the efficiency of development and deployment cycles. Lastly, it will provide valuable feedback and lessons learned from the evaluation of these implementations, contributing to future improvements in the system.

\section{Document Outline}

Introduction
State of the Art
Work Plan