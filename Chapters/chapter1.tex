%!TEX root = ../template.tex
%%%%%%%%%%%%%%%%%%%%%%%%%%%%%%%%%%%%%%%%%%%%%%%%%%%%%%%%%%%%%%%%%%%
%% chapter1.tex
%% NOVA thesis document file
%%
%% Chapter with introduction
%%%%%%%%%%%%%%%%%%%%%%%%%%%%%%%%%%%%%%%%%%%%%%%%%%%%%%%%%%%%%%%%%%%

\typeout{NT FILE chapter1.tex}%

\chapter{Introduction}
\label{cha:introduction}

\prependtographicspath{{Chapters/Figures/Covers/}}

\section{Context and Motivation}
    
The KLx Employee Management System, AnA, is a tool for the automation of the IGA process, which manages more than 5000 employees from the Credit Agricole group. The software was developed by an external editor named Netwrix and then bought on-premises by the Leasing and Factoring branch of the group. The software's end users are the managers from the global Human Resources team and was deployed to production in 2020.

Since then, the KLx team in Lisbon has worked to maintain the application and upgrading it to meet the needs and requirements that are requested by the Functional team in France, which, in turn, takes its orders from the human resources department.

However, the complexity of the system has increased over time. The integration of more departments into the software and the changes taking place in the environment were aggravating. This made all changes needed more difficult and the cost to maintain the system increases. The work is motivated by a realization that the quality of the code, along with its engineering process, determines the efficiency and effectiveness of the system.

In these situations, the theories and methodologies of Software Quality becomes essential. The intention is to identify some critical points of improvement in the solution and in the development workflow. By using making use of these methods, we intend to achieve a better software and developing system.

\section{Problem Definition}

Besides all the mentioned problems, one thing that marks the codebase is the fact that before the KLx team were created, the objective of the CA was to hire the software editor to do the configuration of the application based on the requirements of the Human Resources. But this was not possible due to the lack of understanding on how the IGA workflow would work, leading to a model misspecification. 

Additionally, at the present, the source code consists of about 20,000 lines spread over 200 files that are not maintained under any versioning control system. The codebase and it's backups are stored in shared servers where the different environments of the application runs. This workflow has led to mistakes that could have been avoided with the safety and organisation provided by a good development model.

More on, a brief review of the source code also revealed some recurrent issues related to the quality of the code, in this case, code duplication. Since there are pretty frequent changes in business, for each change, the time lost in complexity increase every iteration. These, in turn, have introduced several bad engineering practices into the developing system.

The problems our team experiences are related to bad practices an a absence of a Software Quality orientation. Finding and implementing these practices will thus not only improve the quality of the code but also the entire delivery process. In the future, effective ways need to be found to solve these issues.

\section{Document Outline}

This document is structured into five main sections. Below is a brief summary of each section:

\begin{enumerate}
    \item \textbf{Introduction}: This section introduces the overall theme of the research, explaining its importance and the fundamental objectives that guide the study.
    
    \item \textbf{Background}: Provides a comprehensive background to lay the foundational knowledge necessary for understanding the research topic. It sets the context within which the research questions have been framed.
    
    \item \textbf{Related Work}: Discusses existing research and projects that are directly relevant to the study. This comparison helps to highlight the unique aspects of the current research and situates it within the existing academic literature.
    
    \item \textbf{Preliminary Work}: Describes any initial experiments, pilot studies, or preparatory research conducted to validate the feasibility of the research approach and methodologies.
    
    \item \textbf{Work Plan}: Outlines the methodology, planned activities, and timeline for the research. This section is crucial for detailing the steps that will be followed to achieve the research goals and includes milestones and expected outcomes.
\end{enumerate}
