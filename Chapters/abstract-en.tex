%!TEX root = ../template.tex
%%%%%%%%%%%%%%%%%%%%%%%%%%%%%%%%%%%%%%%%%%%%%%%%%%%%%%%%%%%%%%%%%%%%
%% abstract-en.tex
%% NOVA thesis document file
%%
%% Abstract in English([^%]*)
%%%%%%%%%%%%%%%%%%%%%%%%%%%%%%%%%%%%%%%%%%%%%%%%%%%%%%%%%%%%%%%%%%%%

\typeout{NT FILE abstract-en.tex}%

O sistema integrado de gerenciamento de funcionários e aplicativos KLx, também conhecido como AnA, fornece a base para a automação de processos de IGA para mais de 5000 funcionários no Grupo Credit Agricole. Este software local é desenvolvido pela Netwrix e gerenciado pela equipe KLx em Lisboa. Com o tempo, tornou-se mais complexo devido às integrações aceleradas de múltiplos departamentos e ambientes em mudança. Juntamente com essa complexidade, devido à falta de controle de versão e à implantação ineficaz de código, tanto a taxa de entrega quanto a qualidade do código diminuíram.

O foco desta pesquisa é melhorar a eficiência e a eficácia do sistema através da aplicação de metodologias de qualidade de software. As soluções propostas que serão implementadas incluem uma rigorosa refatoração de código para tornar o código fácil de ler e manter, e o uso do GIT para controle de versão. Espera-se que essas medidas reduzam a duplicação de código, simplifiquem o processo de entrega e alcancem uma base de código sólida e organizada, que aumente dramaticamente a produtividade da equipe e a confiabilidade do sistema.

\keywords{
  Software Quality \and
  Identity Governance and Administration \and
  Version Control \and
  Code Refactoring
}
