%!TEX root = ../template.tex
%%%%%%%%%%%%%%%%%%%%%%%%%%%%%%%%%%%%%%%%%%%%%%%%%%%%%%%%%%%%%%%%%%%%
%% abstract-en.tex
%% NOVA thesis document file
%%
%% Abstract in English([^%]*)
%%%%%%%%%%%%%%%%%%%%%%%%%%%%%%%%%%%%%%%%%%%%%%%%%%%%%%%%%%%%%%%%%%%%

\typeout{NT FILE abstract-en.tex}%

The KLx employee and application management system, known as AnA, is integral to automating Identity Governance and Administration (IGA) processes for over 5000 employees within the Credit Agricole Group. This on-premises software, developed by Netwrix and managed by the KLx team in Lisbon, has become increasingly complex due to rapid integration of various departments and evolving environments. This complexity, coupled with a lack of version control and inefficient deployment processes, has led to a decline in delivery rate and code quality.

The primary aim of this research is to enhance the efficiency and effectiveness of the system by implementing software quality methodologies. The proposed solutions include meticulous refactoring of the codebase to improve readability and maintainability, implementing GIT for version control, and automating deployment with GITLAB. These interventions are expected to reduce code duplication, streamline the delivery process, and ensure a robust, organised codebase, thereby significantly enhancing team productivity and system reliability.

\keywords{
  Software Quality \and
  Identity Governance and Administration \and
  Version Control \and
  Code Refactoring
}
