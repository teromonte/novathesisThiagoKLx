%!TEX root = ../template.tex
%%%%%%%%%%%%%%%%%%%%%%%%%%%%%%%%%%%%%%%%%%%%%%%%%%%%%%%%%%%%%%%%%%%%
%% abstract-en.tex
%% NOVA thesis document file
%%
%% Abstract in English([^%]*)
%%%%%%%%%%%%%%%%%%%%%%%%%%%%%%%%%%%%%%%%%%%%%%%%%%%%%%%%%%%%%%%%%%%%

\typeout{NT FILE abstract-en.tex}%

The KLx employee and application management system is a tool for automating Identity Governance and Administration (IGA) processes. The main objective of this thesis is to contribute to improve the quality of the code, proposing optimizations in some functions/ blocks of code and also extend its capabilities to connect with other applications of the group, including the implementation of all/some of the proposals. The thesis will propose and give the necessary information to implement some enhancements and extensions, such as: Refactoring some sets of the codebase in some module to eliminate code smells and improve readability and maintainability; Implement version control with GIT to track and manage code changes; Automate the deployment of the application in the several environments used with GITLAB; Develop new features and integrations, demanded by the stakeholders, with other applications using technologies such as PowerShell, \Verb!C#!, XML, JSON, and SQL.

Concerning its contents, the abstracts should not exceed one page and may answer the following questions (it is essential to adapt to the usual practices of your scientific area):

\begin{enumerate}
  \item What is the problem?
  \item Why is this problem interesting/challenging?
  \item What is the proposed approach/solution/contribution?
  \item What results (implications/consequences) from the solution?
\end{enumerate}

% Palavras-chave do resumo em Inglês
% \begin{keywords}
% Keyword 1, Keyword 2, Keyword 3, Keyword 4, Keyword 5, Keyword 6, Keyword 7, Keyword 8, Keyword 9
% \end{keywords}
\keywords{
  One keyword \and
  Another keyword \and
  Yet another keyword \and
  One keyword more \and
  The last keyword
}
