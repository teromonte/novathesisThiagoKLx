\typeout{NT FILE chapter4.tex}%

\chapter{Preliminary work}
\label{cha:Preliminary work}

\prependtographicspath{{Chapters/Figures/Covers/}}

\section{Preliminary Work}

\subsection{Refactoring the Codebase}

\subsubsection{Request 69168 - Restructuring the Application Profiles and Permissions}

The objective of this task is to restructure the application profiles and permissions according to a new matrix, aiming to reduce code duplication in dozens of classes and address various code smells. This refactoring effort is driven by the findings of a recent code audit, which identified significant software quality issues. By reorganizing the profiles and permissions, we anticipate improvements in code maintainability and readability.

Steps Involved

     Analyze the current structure of application profiles and permissions.
     Design a new matrix that optimizes the structure and reduces redundancy.
     Refactor the existing codebase to align with the new matrix.
     Test the refactored code to ensure functionality remains intact.


\subsubsection{Request 67857 - Correcting Anomalies in the Permission Catalog}

This task focuses on correcting anomalies identified in the permission catalog for end users. Addressing these anomalies is crucial for ensuring the proper functioning of the permission system and enhancing overall user experience.

\textbf{Steps Involved:}

     Review and document the identified anomalies.
     Investigate the root causes of these anomalies.
     Implement code changes to correct the anomalies.
     Validate the corrections through extensive testing.


\subsection{Implementing Version Control with GIT}

The implementation of version control using GIT is a multi-phase process that ensures efficient tracking and management of code changes. This section outlines the steps already accomplished and the phases that are yet to be executed.

\textbf{Phases Already Accomplished:}

     Code clean-up in preparation for migration.
     Repository creation.
     Script variabilization.
     AD String mapping.
     Adaptation of scripts for different environments.
     Definition of access, extensions, and variabilization.
     Team training on the new version control system.


\subsection{Automating Deployment with GITLAB}

The automation of the deployment process using GITLAB aims to streamline the deployment across various environments. This task involves identifying inefficiencies in the current process and transitioning it from a local environment to a cloud-based one, integrated with a versioning tool.

\textbf{Steps Involved:}

     Document and analyze the current deployment process.
     Identify and address shortcomings in the current process.
     Evaluate the feasibility of using native development tools for automation.
     Transition the deployment process to a cloud-based environment.
     Integrate the deployment process with the versioning tool.


\subsection{Developing New Features and Integrations}

The development of new features and integrations is driven by stakeholder demands, utilizing technologies such as PowerShell, C\#, XML, JSON, and SQL. Below are examples of specific integration requests.

\subsubsection{Request 68123 - Integration of Application L684 SAS Serveur}

This request involves integrating the application L684 SAS Serveur through both manual and AD provisioning.

\textbf{Steps Involved:}

     Analyze the requirements for the integration of L684 SAS Serveur.
     Develop the necessary scripts for manual and AD provisioning.
     Test the integration to ensure compatibility and functionality.


\subsubsection{Request 68184 - Integration of Application L716-SIMPLISSIMO}

This request focuses on the manual provisioning of the application L716-SIMPLISSIMO.

\textbf{Steps Involved:}

     Gather requirements for the integration of L716-SIMPLISSIMO.
     Develop and implement manual provisioning scripts.
     Validate the integration through rigorous testing.


\subsubsection{Request 68360 - Integration of Application L099A-Tables START}

This task involves the manual provisioning of the application L099A-Tables START.

\textbf{Steps Involved:}

     Identify the requirements for the integration of L099A-Tables START.
     Create and execute manual provisioning scripts.
     Conduct testing to ensure the integration meets all specified requirements.



